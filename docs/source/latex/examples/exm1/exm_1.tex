

\documentclass[11pt]{report}
\usepackage{graphicx}
\usepackage{fontspec}
\usepackage{xltxtra}
\usepackage{layouts}
\usepackage{textcomp}
\usepackage[nonumeralsign, stigma]{xgreek}
\setmainfont{Lato}        % ή GFS Artemisia
% Tinos
% Lato
% Fira
% Georgia Pro (++ πολύ καλή)
% Noto Serif
% Roboto Slab

\setsansfont{Lato}

\setmonofont{Liberation Mono}

\setlength{\parskip}{\baselineskip}%
\setlength{\parindent}{0pt}%
\usepackage{color}
\usepackage[switch, displaymath, mathlines]{lineno}
\renewcommand\linenumberfont{\bfseries\tiny\sffamily}
\renewcommand\thelinenumber{\color{red}\arabic{linenumber}}
\linenumbers

\begin{document}


\chapter{Βασικά στοιχεία}
	
\section{First section}

\begin{figure}
	\oddpagelayoutfalse
	\twocolumnlayouttrue
	\pagediagram
	\caption{Left-hand two-column page layout parameters} \label{fig:pplt}
\end{figure}

Some text. πάνω \& κάτω
	
$\langle x \rangle$

\begin{tabular}{lllc}
	ID          & \multicolumn{2}{c}{Όνομα} & Ηλικία \\
	\hline
	978-0-03701 & Παπαδόπουλος & Νίκος      & 28  \\
	654-2-02262 & Δικαίου      & Μαρίνα     & 35  
\end{tabular}

\begin{tabular}{lccl} 
	\textit{ID}       &\multicolumn{2}{c}{\textit{Name}} &\textit{Age} \\
	\hline  
	978-0-393-03701-2 &O'Brian &Patrick                  &55           \\
	...
\end{tabular}

\begin{tabular}{l|r@{--}l} 
	\multicolumn{1}{c}{\textsc{Period}} & \multicolumn{2}{c}{\textsc{Span}} \\ \hline
	Baroque          &1600              &1760         \\
	Classical        &1730              &1820         \\
	Romantic         &1780              &1910         \\
	Impressionistic  &1875              &1925
\end{tabular}

{\slshape Αν και το το συγκεκριμένο περιβάλλον δε θεωρείται παρωχημένο (obsolete), συνήθως για το σκοπό αυτό χρησιμοποιούνται ισχυρότερα συστήματα δημιουργίας γραφικών όπως τα TikZ, PSTricks, MetaPost και Asymptote. Μην ξεχνάτε ότι στο αγγλοσαξωνικό σύστημα αρίθμησης (που είναι διεθνές), τα σύμβολα τελεία ``.`` και κόμμα ``,`` έχουν ακριβώς την αντίθετη χρήση από το ελληνικό.}

\newpage


	
	
	\begin{sloppypar}
		Αν και το το συγκεκριμένο περιβάλλον δε θεωρείται παρωχημένο (obsolete), συνήθως για το σκοπό αυτό χρησιμοποιούνται ισχυρότερα συστήματα δημιουργίας γραφικών όπως τα TikZ, PSTricks, MetaPost και Asymptote. Μην ξεχνάτε ότι στο αγγλοσαξωνικό σύστημα αρίθμησης (που είναι διεθνές), τα σύμβολα τελεία ``.`` και κόμμα ``,`` έχουν ακριβώς την αντίθετη χρήση από το ελληνικό.
	\end{sloppypar}

\pagebreak

\begin{center}
	\# \$ \% \& \{ \} \_ \~{} \^{} \textbackslash \\
	\verb!# $ % & { } _ ~ ^ \!
\end{center}

text and [text]

\the\baselineskip

A linear function is a function of the form \[ y = mx + c \]

\[ 
\left[  \frac{ N } { \left( \frac{L}{p} \right)  - (m+n) }  \right]
\]
	
	
	The wave equation for \( u \) is
	\begin{displaymath}
	\frac{\partial^2u}{\partial t^2} = c^2\nabla^2u
	\end{displaymath}
	where \( \nabla^2 \) is the spatial Laplacian and \( c \) is constant.
	
	
	
	The variable \(\mathbf{text}x\) is transformed by the function \(f(x)\).
	
	contrast x+y with \( x+y \) and \(\displaystyle \sum_{n=0}^\infty x_n\)
	
	\begin{tabular}{r|cc}
		\textsc{Name}  &\textsc{Series}  &\textsc{Sum}  \\  \hline
		Arithmetic     &\( a+(a+b)+(a+2b)+\cdots+(a+(n-1)b) \)
		&\( na+(n-1)n\cdot\frac{b}{2}\)  \\
		Geometric      &\( a+ab+ab^2+\cdots+ab^{n-1} \)
		&\(\displaystyle a\cdot\frac{1-b^n}{1-b}\)  \\
	\end{tabular}
	
	\section{Second section}
	
	
	Η μεταβλητή \begin{math} y = f(x) \end{math} εξαρτάται από τη μεταβλητή \begin{math} x \end{math}.
\end{document}